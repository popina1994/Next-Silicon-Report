\documentclass[12pt]{article}

\usepackage[utf8]{inputenc}
\usepackage[T1]{fontenc}
\usepackage{lmodern}
\usepackage{amsmath, amssymb}
\usepackage{graphicx}
\usepackage{hyperref}
\usepackage{geometry}
\usepackage{fancyhdr}
\usepackage{enumitem}
\usepackage{listings}
\usepackage{xcolor}
\usepackage{algorithm2e}
\usepackage{algorithmicx}
\lstset{
  language=C++,
  basicstyle=\ttfamily\small,
  keywordstyle=\color{blue}\bfseries,
  stringstyle=\color{red},
  commentstyle=\color{gray},
  morecomment=[l][\color{magenta}]{\#},
  numbers=left,
  numberstyle=\tiny\color{gray},
  stepnumber=1,
  numbersep=10pt,
  backgroundcolor=\color{white},
  showspaces=false,
  showstringspaces=false,
  emph={int,float, double},
  frame=single,
  tabsize=4,
  breaklines=true,
  emphstyle=\color{purple},
  escapeinside={*@}{@*}
  }
% escapechar=|,


\geometry{a4paper, margin=1in}
\pagestyle{fancy}
\fancyhf{}
\rhead{\thepage}
\lhead{Dorde Zivanovic}

\title{Next Silicon: CM Home Assignment}
\author{Dorde Zivanovic \\ \small{LinScale}}
\date{\today}

\begin{document}

\maketitle
\tableofcontents
\newpage

\section{Introduction}
This report contains the responses to the tasks stated in the home project pdf.
It is accompanied with the repository that contains reproducible solutions with the installation instructions alongside: experiments and tests according to the task requirements.
\section{Code Analysis and Documentation}
In this section we analyze the existing code shown in Algorithm \ref{alg:sine_existing}.

\subsection{The Existing Implementation: Code Drawbacks}
The code is written for C and follows the following bad practices:
\begin{enumerate}
    \item Reusing the variable multiple times,  \texttt{(float)M\_PI} and \texttt{2.0f * (float)M\_PI}, (lines \ref{alg:sine_ex:ln:fmodf}, \ref{alg:sine_ex:ln:xGreater}, \ref{alg:sine_ex:ln:xLess}, \ref{alg:sine_ex:ln:xLess}, \ref{alg:sine_ex:ln:xMinus}, \ref{alg:sine_ex:ln:xPi} of Algorithm \ref{alg:sine_existing});
    \item Not using auto in order to automatically deduce types since the results of all the statements are known;
    \item Cleaning if loop to be more understandable: \texttt{fmodf} returns the result in the range $(-2 \pi, 2 \pi)$. Then, now it is obvious that one checks whether the number is outside of the range $[-\pi, \pi]$, and then updates \textbf{x} for $2 \pi$ period, so the method works from the number in the range $[-\pi, \pi]$;
    \item Adding more verbosity ();
    \item Reusing variable names -> more verbose names should be used in order to improve readability of the code. The c ompiler will optimize for the least number of variables/registers to be used;
    \item Renaming function names and migrating these functions to the corresponding headers and sources that would contain the custom maths functions.
\end{enumerate}


\begin{algorithm}
    \caption{Algorithm with Code Listing}
    \begin{algorithmic}[1]
    \State \textbf{Input:} A float (IEEE-754) number
    \State \textbf{Output:} A float (IEEE-754) sine value of this number computed using Taylor Series.
    \State \textbf{Steps:}
\begin{lstlisting}[numbers=left]
float fp32_custom_sine(float x)
{
    x = fmodf(x, 2.0f * (float)M_PI);  *@\label{alg:sine_ex:ln:fmodf}@*
    if (x > (float)M_PI) *@\label{alg:sine_ex:ln:xGreater}@*
        x -= 2.0f * (float)M_PI; *@\label{alg:sine_ex:ln:xMinus}@*
    else if (x < -(float)M_PI) *@\label{alg:sine_ex:ln:xLess}@*
        x += 2.0f * (float)M_PI; *@\label{alg:sine_ex:ln:xPi}@*
    float result = 0.0f;
    float term = x;
    float x_squared = x * x;
    int sign = 1;
    for (int n = 1; n <= 7; n += 2)
    {
        result += sign * term;
        sign = -sign;
        term = term * x_squared;
        term = term / (float)(n + 1);
        term = term / (float)(n + 2);
    }
    return result;
}

\end{lstlisting}
\end{algorithmic}
 \label{alg:sine_existing}
\end{algorithm}

\subsection{Drawbacks: numerical and implementation issues}
Here, I will give state several main drawbacks in terms of the implementation and numerical accuracy. The division by
In the next subsection, I will list the drawbacks related to the method itself.
\subsection{Mathematical Analysis}
\subsection{Failures}
\subsection{Test Plan}

\section{Conclusion}
Summarize your findings or thoughts here.
\cite{apostol1985mathematical}
\bibliographystyle{plain}
\bibliography{refs}

\end{document}
