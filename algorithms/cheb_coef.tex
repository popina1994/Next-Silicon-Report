
\begin{algorithm}
    \caption{Computing Chebyshev Coefficients}
    \begin{algorithmic}[1]
    \State \textbf{Input:} A function \texttt{f}, the number of coefficients to compute \texttt{numCoeffs} and bounds of the interval \texttt{a} and \texttt{b}.
    \State \textbf{Output:} A return vector that contains the computed values.
\begin{lstlisting}[numbers=left]
    std::vector<float> computeChebyshevCoefficients(std::function<float(float)> f, uint32_t numCoeffs, float a, float b) {
        std::vector<float> vCoeffs(numCoeffs, 0.f);

        float bma = 0.5f * (b - a); *@\label{alg:sine_cheb:ln:scaling_back_0}@*
        float bpa = 0.5f * (b + a); *@\label{alg:sine_cheb:ln:scaling_back_1}@*
        for (uint32_t j = 0u; j < numCoeffs; j++) {
            float sum = 0.f;

            for (uint32_t k = 0u; k < numCoeffs; k++) {
                float leftTheta = std::numbers::pi_v<float> * (k + 0.5f) / numCoeffs;
                float rightTheta = leftTheta * j;
                float leftCos = std::cos(leftTheta);
                float rightCos = std::cos(rightTheta);
                sum += f(leftCos * bma + bpa) * rightCos; *@\label{alg:sine_cheb:ln:scaling_back_2}@*
            }
            vCoeffs[j] = sum * 2.0f / numCoeffs;
        }

        return vCoeffs;
    }
\end{lstlisting}
\end{algorithmic}
 \label{alg:cheb_coef}
\end{algorithm}