
\begin{algorithm}
    \caption{Sine using Chebyshev Approximation}
    \begin{algorithmic}[1]
    \State \textbf{Input:} A float (IEEE-754) number and the degree \texttt{chebDegreeN} used in Chebyshev approximation.
    \State \textbf{Output:} A float (IEEE-754) sine value of this number computed using Chebyshev approximation.
\begin{lstlisting}[numbers=left]
    float nextSiliconSineFP32Chebyshev(float x, uint32_t chebDegreeN)
    {
        static constexpr auto PI_F = pi_v<float>;
        static constexpr auto TWO_PI_F = 2 * PI_F;

        float xPiRange = x;
        if (std::abs(xPiRange) >= TWO_PI_F)
        {
            xPiRange = optimizedFModf2Pi(xPiRange);
        }
        if (std::abs(xPiRange) > PI_F)
        {
            xPiRange -=  boost::math::sign(xPiRange) * TWO_PI_F;
        }

        static constexpr auto epsilonFloat = std::numeric_limits<float>::epsilon();
        auto b = xPiRange + xPiRange * (4 * epsilonFloat);
        auto a = xPiRange - xPiRange * (4 * epsilonFloat);
        auto y = (2.0f * xPiRange - a - b) / (b - a); *@\label{alg:sine_cheb:ln:scaling}@*
        auto y2 = 2.f * y;
        auto chebCoeffs = computeChebyshevCoefficients(::sinf, chebDegreeN, a, b);

        float dMPlusTwo = 0.f;
        float dMPlusOne = 0.f;

        // Clenshaw's formula
        for (std::size_t k = chebDegreeN - 1; k > 0; k--) {
            float bCurr = y2 * dMPlusOne - dMPlusTwo + chebCoeffs[k];
            dMPlusTwo = dMPlusOne;
            dMPlusOne = bCurr;
        }

        return  y * dMPlusOne - dMPlusTwo +  chebCoeffs[0] * 0.5;
    }

\end{lstlisting}
\end{algorithmic}
 \label{alg:sine_cheb_opt}
\end{algorithm}