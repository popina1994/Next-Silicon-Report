
\begin{algorithm}
    \caption{Sine using Taylor series: Optimized Method}
    \begin{algorithmic}[1]
    \State \textbf{Input:} A float (IEEE-754) number and the degree \texttt{degreeEnd} used in Taylor series.
    \State \textbf{Output:} A float (IEEE-754) sine value of this number computed using Taylor series.
\begin{lstlisting}[numbers=left]
    float nextSiliconSineFP32Taylor(float x, uint32_t degreeEnd)
    {
        static constexpr auto PI_F = pi_v<float>;
        static constexpr auto TWO_PI_F = 2 * PI_F;
        static constexpr auto TAYLOR_DEGREE_START = 3u;
        static constexpr auto TAYLOR_DEGREE_NEXT_INC = 2u;

        if (std::abs(x) == 0.f) { return 0; }

        auto xPiRange = fmodf(x, TWO_PI_F);
        if (std::abs(xPiRange) > PI_F)
        { xPiRange -=  boost::math::sign(xPiRange) * TWO_PI_F; }

        auto result = xPiRange;
        auto term = xPiRange;
        const auto xPiRangeSquared = xPiRange * xPiRange;
        auto sign = 1;
        std::vector<float> vFact(degreeEnd + 1, 0);
        vFact[1] = 1;

        for (auto taylorDegree = TAYLOR_DEGREE_START;
            taylorDegree <= degreeEnd;
            taylorDegree += TAYLOR_DEGREE_NEXT_INC)
        { vFact[taylorDegree] = vFact[taylorDegree - 2] * (taylorDegree - 1) * taylorDegree;}

        for (auto taylorDegree = TAYLOR_DEGREE_START;
            taylorDegree <= degreeEnd;
            taylorDegree += TAYLOR_DEGREE_NEXT_INC)
        {
            sign = -sign;
            term = term * xPiRangeSquared;
            auto termAdd = term / float(vFact[taylorDegree]);
            if (sign == 1)
            {
                result += termAdd;
            }
            else
            {
                result -= termAdd;
            }
        }

        return result;
    }

\end{lstlisting}
\end{algorithmic}
 \label{alg:tay_optimized}
\end{algorithm}